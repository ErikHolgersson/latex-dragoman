
\documentclass{article}






\usepackage[utf8]{inputenc}




\usepackage[T1]{fontenc}




\usepackage{lmodern}

%

%
\usepackage[ngerman]{babel}




\usepackage{amsmath}






\title{Ein Testdokument / A test document}




\author{Otto Normalverbraucher}




\date{5. Januar 2004 / January 5, 2004}




\begin{document}






\maketitle



\tableofcontents



\section{Einleitung / introduction}


Hier kommt die Einleitung. Ihre Überschrift kommt
automatisch in das Inhaltsverzeichnis.




Here comes the introduction. Your heading comes
automatically in the table of contents.


\subsection{Formeln / formulas}






\LaTeX{}
 ist auch ohne Formeln sehr nützlich und
einfach zu verwenden. Grafiken, Tabellen,
Querverweise aller Art, Literatur- und
Stichwortverzeichnis sind kein Problem.

Formeln sind etwas schwieriger, dennoch hier ein
einfaches Beispiel. Zwei von Einsteins
berühmtesten Formeln lauten:

 is very useful and
easy to use even without formulas. Graphics, tables,
cross-references of all kinds, bibliography and
index are no problem.

Formulas are a bit more difficult, yet here is a
simple example. Two of Einstein's
most famous formulas are:

\begin{align}

E &= mc^2 

\
m &= \frac{m_0}{\sqrt{1-\frac{v^2}{c^2}}}

\end{align}

Aber wer keine Formeln schreibt, braucht sich
damit auch nicht zu beschäftigen.


But if you don't write formulas, you don't need
to bother with them.

\end{document}



