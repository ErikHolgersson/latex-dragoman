\documentclass

\usepackage
\usepackage
\usepackage
%\usepackage
\usepackage
\usepackage

\title{A test document}
\author
\date{January 5, 2004}
\begin

\maketitle

\chapter{transmission on baseband}\label
\section[ISI-free pulse transmission]{intersymbol interference free pulse transmission}
\subsection{transmission model}

The baseband data transmission system shall be replaced by the system shown in Fig
.\ref The input signal $d(t)$ for the transmit filter can be described in the block diagram shown below.\index is a sequence of Dirac pulses
\begin\sum\infty\infty\delta\Ko
\end
which are weighted by the discrete-time sequence $ATd[k]$, where $d[k]$ denote the data symbols of the source.
For a binary transfer, for example,
\begin\in\Pu
\end
The amplitude $A$ controls the power of the transmitted signal $u(t)$, the
clock period\index $T$ is the transmission speed. The transmit filter
with the impulse response $h_\S(t)$ forms the transmit signal 
\index\index
\begin\begin\leavevmode
\epsfxsize\psfrag\tilde\psfrag\psfragfig\psfragfig\caption\label\end\end
\begin\label\begin\ast\S\
\Big\sum\infty\infty\delta\Big\ast\S\
\sum\infty\infty\S\
\sum\infty\infty\Pu
\end\end
The relationship between transmission pulse $g(t)$\index and impulse response of the transmit filter $h_\S(t)$ is given by the relation
\begin\label\S\qquad\le\end
given, where $T_g$ is the duration of the impulse $g(t)$ or the
impulse response $h_\S(t)$ of the transmit filter. For the energy of the
transmitting pulse applies
\begin\label\mathcal\int\int\S\int\infty\infty\S\Pu
\end
The received signal at the output of the receive filter\index is denoted by $y(t)$
and consists of the unfiltered useful signal $x(t)$ and
the filtered noise signal\index $\tilde n(t)$
\begin\label\tilde\Pu
\end
The unnoised received signal is calculated by convolving the
input signal with the impulse response of the transmission system
$h(t)$
\begin\label\begin\ast\
\Big\sum\infty\infty\delta\Big\ast\
\sum\infty\infty\end\end
mit
\begin\label\S\ast\K\ast\E\Pu
\end\
Die Fourier transformed impulse response $h(t)$ yields the
transfer function\index of the system $H(f)$, which can be calculated by multiplying the
transfer functions $H_\S(f)$, $H_\K(f)$ and $H_\E(f)$ for
transmission filter, transmission channel\index and receive filter results
\begin\label\Fo\S\K\E\Pu
\end
Gaussian white noise $n(t)$ is additively superimposed on the useful signal $v(t)$ at the input of the receive filter. The abbreviation AWGN is used for this interference.\index (\emph) are commonly used. Their autocorrelation function\index\index (AKF) reads
\begin\label\phi\tau\textrm\tau\frac\delta
\tau\Ko
\end
where $N_0/2$ is the two-sided
noise power density\index\index
\begin\label\Phi\Fo\phi\tau\frac\end
designated. The detector\index makes the decision about the sent symbols and returns the estimated symbols
$ in the symbol clock.\hat d[k]$ off. Wrong decisions are possible with the given
transfer model according to Fig.\ref possible as a result of linear
signal distortion and noise.


\subsection{1st Nyquist criterion of data transmission}\label 
\index 
In this chapter, the influence of linear distortions on the transmitted signal is investigated,
whereby noise is disregarded. First, the question
shall be posed as to what requirements the impulse response $h(t)$, or 
the transfer function $H(f)$, of a distortion-free
transmission system must satisfy. Distortion-free transmission
means that the signal shape of the transmitted signal is preserved.
A proportional change of the signal (amplitude scaling) as well as a signal delay due
to the finite propagation velocity of the electrical or 
electromagnetic signals is allowed. Image\ref shows
such a distortion-free transmission system. The impulse response of this transmission system is a delayed Dirac impulse
.\begin\label\delta\Ko
\end
which is multiplied by the constant $c$. The output signal $x_2(t)$ is then calculated as
\begin\begin\ast\nonumber\
\delta\ast\
\Pu
\end\end
The Fourier transform\index of this relation yields the corresponding demand on the transfer function $H(f)$
\begin\centering
\leavevmode
\epsfxsize\epsfbox\psfragfig\caption\label\end
\begin\label\Fo\int\infty\infty\e\j\pi\frac\textrm\j\pi\Pu
\end
The magnitude of the transfer function of a distortion-free system
must be constant
.\begin\label\Ko
\end
the argument of the transfer function is a linear function of the
frequency
.\begin\label\arg\pi\Pu
\end
The relationship between transfer function $H(f)$ and
attenuation\index $a(f)$ and phase\index is
$b(f)$
\begin\label\textrm\j\Pu
\end
 From this follow the known conditions of a distortion-free
transmission for attenuation
.\begin\label\ln\textrm\Ko
\end
Phase
\begin\label\arg\pi\end
and group runtime\index
\begin\label\gr\frac\pi\frac\Pu
\end
This means that if the attenuation of a transmission system is constant and the phase is linear, then all spectral components of the transmitted signal are attenuated and delayed equally, and the received signal remains undistorted. However, if these conditions are not met within the signal bandwidth, attenuation and phase distortions of the signal result. These distortions are referred to as\emph\index\index\index
summarized. Linear distortions cause transients and therefore affect successive symbols. This phenomenon is called intersymbol interference (ISI).\index Designated.



\paragraph{Example.} Image\ref shows how a rectangular pulse\index $x_1(t)$ of duration $T$ by an RC element with time constant $\tau = RC = T/2$ is distorted. The output pulse $x_2(t)$ spans more than two symbol periods $T$, causing ISI. Analysis shows that neither the damping nor the phase conditions for a distortion-free system are met. The transfer function is
\begin\label\frac\frac\sqrt\pi\tau\textrm\j\arctan\pi\tau\Pu
\end
\begin\begin\leavevmode
\epsfxsize\psfrag\psfrag\psfrag\epsfbox\psfragfig\caption\tau\label\end\end
For the damping and phase, one obtains from this the relationships
\begin\label
 a( f ) & =\tfrac \cdot \ln \left(1 + (2\pi f\tau)^2\right)\Ko \
  b(f) & =\arctan(2\pi f\tau)\Pu
\end

\end
\end{document}